\documentclass{article}

\usepackage{listings}
\usepackage{indentfirst}
\usepackage{verbatim}
\usepackage{amsmath, amsthm, amssymb}
\usepackage{stmaryrd}
\usepackage{graphicx}
\usepackage{euscript}

\usepackage[utf8]{inputenc}
\usepackage[english,russian]{babel}
\usepackage[T2A]{fontenc}

\lstdefinelanguage{llang}{
keywords={skip, do, while, read, write, if, then, else},
sensitive=true,
%%basicstyle=\small,
commentstyle=\scriptsize\rmfamily,
keywordstyle=\ttfamily\underbar,
identifierstyle=\ttfamily,
basewidth={0.5em,0.5em},
columns=fixed,
fontadjust=true,
literate={->}{{$\to$}}1
}

\lstset{
language=llang
}

\newcommand{\aset}[1]{\left\{{#1}\right\}}
\newcommand{\term}[1]{\mbox{\texttt{\bf{#1}}}}
\newcommand{\cd}[1]{\mbox{\texttt{#1}}}
\newcommand{\sembr}[1]{\llbracket{#1}\rrbracket}
\newcommand{\conf}[1]{\left<{#1}\right>}
\newcommand{\fancy}[1]{{\cal{#1}}}

\newcommand{\trule}[2]{\frac{#1}{#2}}
\newcommand{\crule}[3]{\frac{#1}{#2},\;{#3}}
\newcommand{\withenv}[2]{{#1}\vdash{#2}}
\newcommand{\trans}[3]{{#1}\xrightarrow{#2}{#3}}
\newcommand{\ctrans}[4]{{#1}\xrightarrow{#2}{#3},\;{#4}}
\newcommand{\llang}[1]{\mbox{\lstinline[mathescape]|#1|}}
\newcommand{\pair}[2]{\inbr{{#1}\mid{#2}}}
\newcommand{\inbr}[1]{\left<{#1}\right>}
\newcommand{\highlight}[1]{\color{red}{#1}}
\newcommand{\ruleno}[1]{\eqno[\textsc{#1}]}
\newcommand{\inmath}[1]{\mbox{$#1$}}
\newcommand{\lfp}[1]{fix_{#1}}
\newcommand{\gfp}[1]{Fix_{#1}}

\newcommand{\NN}{\mathbb N}
\newcommand{\ZZ}{\mathbb Z}

\begin{document}

\section{Релизация генерирующих расширений в type-driven форме}

Пусть программа $P$ принимает аргументы $x$ и $y$; результат её работы над ними
обозначим $P(x, y)$. Пусть $a$~--- какое-то значение аргумента $x$. Программа
$S^{x=a}_P$ называется \emph{специализацией} $P$ для $x=a$, если

$$
\forall b \;\; S^{x=a}_P(b)=P(a, b)
$$

Программа $G^x_P$ называется \emph{генерирующим расширением} (generating extension) для
программы $P$, если выполнено следующее соотношение:

$$
\forall a \;\; G^x_P(a)=S^{x=a}_P
$$

Иными словами, генерирующее расширение по значению первого аргумента для $P$ порождает
её специализацию на это значение. 

Генерирующее расширение может быть написано либо руками, либо получено с помощью общих
механизмов смешанных вычислений.

В рамках данной темы предполагается разработать способ автоматизированного получения
генерирующих расширений на основе аннотаций типов для Haskell. Идея заключается в том, чтобы,
имея описание для $P$ (видимо, на некотором DSL комбинаторного типа) получать из этого описания
генерирующее расширение, просто меняя аннотацию соответствующего аргумента. Например:

\begin{verbatim}
  p :: Int -> Int -> Int
  p x y = x + y
\end{verbatim}

Поменяем аннотацию у \verb|p|:

\begin{verbatim}
  p :: Int -> Dyn Int -> Text Int
  p x y = x + y
\end{verbatim}

Теперь, применив \verb|p| к 3, получим \emph{текст} генерирующего расширения.

Некоторая литература по теме:

\begin{enumerate}
\item Neil Jones, Carsten Gomard, Peter Sestoft. Partial evaluation and automatic program generation // \url{https://www.itu.dk/~sestoft/pebook/jonesgomardsestoft-a4.pdf}
\item Jaques Carrette, Oleg Kyseliov, Chung-chien Chan. Finally Tagless, Partially Evaluated. Tagless Staged Interpreters for Sumpler Typed Languages // \url{http://okmij.org/ftp/tagless-final/JFP.pdf}
\end{enumerate}

\section{Генерация реляционных программ для решения логических задач}

Реляционное программирование~--- это подход, при котором программы описываются не как функции, а как отношения. Это позволяет
выполнять их ``в разные стороны''~--- например, по результатам получать аргументы. Реляционная спецификация сложения,
таким образом, одновременно является описанием вычитания, умножения~--- деления и т.д.

С точки зрения классификации подходов реляционное программирование можно отнести к тому же классу, что и логическое. Поэтому
одно из возможных примененией~--- это построение решателей логических задач по полностью декларативному описанию.

В данной теме предлагается изучить DSL реляционного программирования miniKanren и его реализацию для языка OCaml, научиться
на нем писать спецификации логических задач, а потом научиться генерировать \emph{экземпляры} этих описаний для
задач разного размера (например, описать генератор задачи о ферзях для любого количества и ферзей и 
любого размера доски). Возможно, что для того, чтобы это сделать безопасным способом, текущих примитивов не хватит.

Некоторая литература по теме:

\begin{enumerate}
\item Язык реляционного программирования miniKanren: \url{http://minikanren.org}; там же
можно почитать про ``канонические'' реализации решателей;
\item Реализация оного для OCaml: \url{https://github.com/dboulytchev/OCanren};
\item Её короткое описание: \url{http://oops.math.spbu.ru/papers/ocanren.pdf}.
\end{enumerate}

%\section{Комбинаторы для кодогенерации}

%\section{Генерация кода и разрешение ограничений}

\section{First-class unification для Haskell}

Унификация термов $p$ и $q$~--- это такая подстановка $\theta$, что $p\theta=q\theta$. Унификация играет
чрезвычайно важную роль в логическом программировании. В рамках данной темы предлагается
попробовать реализовать first-class унификацию для языка Haskell~--- например, научиться по терму
строить задачу унификации, которая дальше могла бы ``путешествовать'' по структурам данных и 
применяться к другим термам.

Некоторая литература по теме:

\begin{enumerate}
\item Franz Baader, Wayne Snyder. Unification Theory (первые две главы): \url{http://www.cs.bu.edu/~snyder/publications/UnifChapter.pdf};
\item Mark Tullsen. First Class Patterns: \url{http://citeseerx.ist.psu.edu/viewdoc/summary?doi=10.1.1.37.7006};
\item Manuel M. T. Chakravarty , Gabriel C. Ditu , Roman Leshchinskiy. Instant Generics: Fast and Easy: \url{http://citeseerx.ist.psu.edu/viewdoc/summary?doi=10.1.1.150.1033}.
\end{enumerate}

\end{document}

